\chapter{通过 RNA-Seq 数据估计转录本的长度}
\label{chap-lenest}

\section{介绍}
在这里我们介绍一个能够通过 RNA-Seq 数据在没有已知基因注释的情况下估计一个转录本的长度的统计方法. 

对于目前大部分现有的 RNA-Seq 数据, 
我们都无法直接从数据中直接获得其所包含的转录本在基因组上的 5' 段和 3' 段的位置. 
但是在 RNA-Seq 实验中为了能够测量每一个转录本的表达量, 
我们在计算其表达量时均需要使用到该转录本的长度 
\cite{mortazavi2008mapping, Jiang15042009, cufflinks.2010}. 
在使用基因组及其所对应的基因注释的情况下 (例如 RefSeq \cite{_refseq}), 
在分析时可以直接采用注释中的基因位置得出各转录本的长度. 
在没有基因注释, 或者在没有基因组参考序列的情况下, 
如果不采用更加复杂的技术, 如 RNA-PET \cite{Fullwood01042009} 技术, 
则需要通过 RNA-Seq 数据估计出转录本的长度. 对于现有的大部分 RNA-Seq 数据, 
仍然没有对应的 RNA-PET 数据用语确定各转录本的 5' 和 3' 段. 
所以对于通过 RNA-Seq 数据直接估计转录本的长度是有需求的. 

\section{理论分析}
在通过 RNA-Seq 数据估计转录本的时候我们对 RNA-Seq 数据做如下的假设以简化模型: 
\begin{itemize}
\item 所有的读段在原转录本上的位置的分布都是独立的

\item 所有的读段的长度都是相同的, 为 $R$ bp 长

\item 读段的 5' 段位置 (以下称之为起始位置) 在原转录本上的分布是均匀的
\end{itemize}

\begin{thm}
假设 $X_1$, $X_2$, ..., $X_N$ 是来自定义在 
${1, 2, ..., L}$, $L \in \mathbb{Z}^+$ 的均匀分布, 
即 
\[
P(X = i) =  \begin{cases}
\frac{1}{L} & \text{ 当 } 1 \leq i \leq L \\
0 & \text{ 其余情况 }
\end{cases}
\]
则对于 $L$ 的最小方差无偏估计 $\hat{L}$ 为
\begin{equation}
\hat{L} = 
\end{equation}
其中 $X_{(N)} = \max_{1 \leq i \leq N} X_i$. 
\end{thm}

\begin{thm}
假设
\end{thm}

\begin{thm}
假设
\end{thm}

\section{仿真分析}

\section{实际 RNA-Seq 数据分析}


