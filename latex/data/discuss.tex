\chapter{讨论}
在这一章中我们对 \ref{chap-lenest} 中通过 RNA-Seq 数据估计转录本的长度的方法, 
\ref{chap-rna-seq-nonunif} 中对于 RNA-Seq 数据分布的不均匀性的描述, 
\ref{chap-rna-seq-nonunif} 中真核生物 RNA-Seq 数据估计基因的转录组问题是 NP 难的阐述, 
以及现有的 RNA-Seq 数据分析方法, 
和 RNA-Seq 数据分析方法的未来的发展方向进行讨论. 

\section{通过 RNA-Seq 数据估计转录本的长度}
在 \ref{chap-lenest} 的分析中我们只对单端等长读段测序数据进行了考虑, 
对于双端数据或者长度不均的读段均未作具体分析. 
在实际应用中可以只对双端数据考虑等长的在原转录本的 5' (或者 3') 端, 
或者长度不均的读段考虑等长的在原转录本的 5' (或者 3') 端, 
从而可以用 \ref{chap-lenest} 中的方法进行分析. 

此外, \ref{chap-lenest} 中所介绍的方法并没有考虑 RNA-Seq 
数据中读段分布受转录本序列以及转录本长度等因素造成的分布不均匀性. 
另外, \ref{chap-lenest} 中所介绍的方法也未考虑读段分布的非独立性. 
这些问题都是需要进一步继续的研究来解决的. 

\section{通过真核生物 RNA-Seq 数据估计基因的转录组}
根据 \ref{chap-rna-seq-nphard} 中的讨论, 
我们知道通过目前的 RNA-Seq 技术从 RNA-Seq 数据中直接估计转录组中转录本的组成在计算上是比较困难的. 
同样的问题在通过测序数据估计单体型 (haplotype) 
\cite{Li_Kim_Waterman_2004, Xing_Jordan_Sharan_2007} 时也是同样存在的 \cite{1668028}. 

随着测序技术的发展, 
RNA-PET \cite{Fullwood01042009} 技术可以对直接用来确定转录本的 5' 端和 3' 端, 
另外更长的测序读段 (例如 PacBio 研发的单分子测序技术 \cite{hybrid.rna.seq.2012})
可以帮助我们在测序数据得到后不需要使用类似 \ref{chap-rna-seq-nphard} 
给出的复杂的方法就能简单地直接从数据中确定转录组由哪些转录本组成. 

\section{现有 RNA-Seq 数据处理方法中存在的一些问题}
目前已有的用于 RNA-Seq 数据辨识基因的剪切异构体和估计表达量的方法 
(\ref{intro-rna-seq-tools-summary}) 仍然存在不足. 
其中包括: 
\begin{itemize}
\item 辨识基因的剪切异构体复杂度高 (真核生物 RNA-Seq 数据)

\item 当有两个转录本其中一个转录本的一端位于另一个转录本的中间时 
(图 \ref{disc-human-gene-alternative-start-1}), 
目前没有有效的方法通过 RNA-Seq 数据对这两个转录本进行区分

\item 对于原核生物, 尤其是原核生物中的操纵子, 的分析方法仍然没有较规范的分析方法 
\cite{mcclure2013computational} (图 \ref{disc-bacteria-operons-1})

\item 估计转录本表达量时仍无系统的方法处理 RNA-Seq 数据中读段分布的不均性, 
以及转录本序列的组成带来的误差 \cite{oshlack2009transcript, jones2012new} 
\end{itemize}

\begin{figure}[!t]
\centering
\includegraphics[width=\textwidth]{figures/disc/disc-human-gene-alternative-start-1.png}
\caption{人的基因组中出现的一个转录本的一端位于另一个转录本的中间的现象}
\label{disc-human-gene-alternative-start-1}
\end{figure}

\begin{figure}[!t]
\centering
\includegraphics[width=\textwidth]{figures/disc/disc-bacteria-operons-1.png}
\caption{\onlinecite{giannoukos2012efficient} 中的 RNA-Seq 数据, 
其中可以看见多个基因构成的操纵子}
\label{disc-bacteria-operons-1}
\end{figure}

\section{未来 RNA-Seq 数据的定量分析方法的工作}

\subsubsection{RNA-PET 测序技术} %% 序列比对, 拼装

\subsubsection{更长的测序读段} %% 序列比对, 拼装

\subsection{原核生物}
原核生物的 RNA-Seq 数据分析对于对于我们研究微生物中的基因表达是十分有意义的. 
此外, 近几年来兴起的宏转录组 (metatranscriptome) 的研究也帮助了我们对于微生物群落有了更为深入的了解
\cite{gilbert2008detection, urich2008simultaneous, gifford2010quantitative, 
helbling2011activity, mason2012metagenome, huson2011integrative, 
lesniewski2012metatranscriptome}. 

\nocite{sorek2009prokaryotic}

与真核生物不同, 原核生物中多个基因会被同时翻译到一个转录本中形成一个操纵子 
(图 \ref{e.coli.lactose.operon}). 
最近几年的研究表明 \cite{MarcGuell11272009, koide2009prevalence}, 
原核生物在不同的环境条件下在基因组的同一段区域可能产生不同的操纵子. 
通过 RNA-Seq 数据辨识原核生物操纵子也将可能成为原核生物 RNA-Seq 数据分析中的重要的一步. 

另外, 在原核生物的转录组中有大量的 rRNA (ribosomal RNA).  
在 RNA-Seq 实验中, 
rRNA 和其他种类的 RNA (例如 mRNA) 相比起来 rRNA 含量可能超过 
90\% \cite{giannoukos2012efficient}. 
此时, rRNA 的含量会对于研究其他种类的 RNA 产生较大的负面影响. 
目前 \onlinecite{giannoukos2012efficient} 中使用了一种新的实验技术, 
可以在测序前减少样本中 rRNA 的含量. 

\begin{figure}[!t]
\centering
\includegraphics[width=\textwidth]{figures/disc/e-coli-lactose-operon.png}
\caption{\textit{E. coli} 中的乳糖操纵子 \cite{shuman2003art}}
\label{e.coli.lactose.operon}
\end{figure}










