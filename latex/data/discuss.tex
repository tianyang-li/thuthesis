\chapter{讨论}

\section{通过 RNA-Seq 数据估计转录本的长度}

\section{通过 RNA-Seq 数据估计基因的转录组}

\section{现有方法中存在的一些问题}

\section{未来 RNA-Seq 数据的定量分析方法的工作}

\subsection{真核生物}

\subsubsection{介绍} %% 需要测的读段更长

\subsubsection{RNA-PET 测序技术} %% 序列比对, 拼装

\subsubsection{PacBio 长读段} %% 序列比对, 拼装

\subsection{原核生物}

\subsubsection{介绍} %% 需要测的读段更长, 从 RNA-Seq 数据中分析操纵子

\subsubsection{与真核生物 RNA-Seq 数据的比较} %% 计算上比原核生物更为简单, 模型相似

\subsubsection{RNA-PET 测序技术} %% 序列比对, 拼装


