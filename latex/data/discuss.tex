\chapter{讨论}

\section{通过 RNA-Seq 数据估计转录本的长度}

\section{通过 RNA-Seq 数据估计基因的转录组}

\section{现有 RNA-Seq 数据处理方法中存在的一些问题}
目前已有的用于 RNA-Seq 数据辨识基因的剪切异构体和估计表达量的方法 
(\ref{intro-rna-seq-tools-summary}) 仍然存在不足. 
其中包括: 
\begin{itemize}
\item 辨识基因的剪切异构体复杂度高

\item 对于原核生物, 尤其是原核生物中的操纵子, 的分析方法仍然没有较规范的分析方法 

\item 估计转录本表达量时仍无系统的方法处理 RNA-Seq 数据中读段分布的不均性, 
以及转录本序列的组成带来的误差 \cite{jones2012new} 
\end{itemize}

\section{未来 RNA-Seq 数据的定量分析方法的工作}

\subsection{真核生物}

\subsubsection{介绍} %% 需要测的读段更长

\subsubsection{RNA-PET 测序技术} %% 序列比对, 拼装

\subsubsection{PacBio 长读段} %% 序列比对, 拼装

\subsection{原核生物}

\subsubsection{介绍} %% 需要测的读段更长, 从 RNA-Seq 数据中分析操纵子

\subsubsection{与真核生物 RNA-Seq 数据的比较} %% 计算上比原核生物更为简单, 模型相似

\subsubsection{RNA-PET 测序技术} %% 序列比对, 拼装


