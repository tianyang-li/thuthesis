\chapter{引言}

\section{RNA-Seq}
\nocite{wang2009rna}

RNA-Seq 是近几年来发展起来的通过深度测序用于研究转录组的一种技术. 
与其他的方法相比起来, RNA-Seq 揭露了生物的转录组中更多的复杂性. 
同时, RNA-Seq 能够更好地研究生物的转录组中各转录本的表达量. 
通过了解细胞中在某种特定条件下转录组中的转录本的组成, 
以及每一个转录本的表达量, 我们可以了解基因上的不同的功能模块, 
进而了解生物的发育过程, 以及疾病与人体之间的关系. 
通过使用 RNA-Seq, 我们已经对编码蛋白质的基因以及它们的剪切异构体有了更为深入的了解. 
此外, RNA-Seq 也帮助我们对于基因上的非编码区域有了更为深入的认识, 
例如 lncRNA (long non-coding RNA).
并且我们对 sRNA (small RNA), microRNA 等也有了更全面的了解. 
\cite{pickrell2010understanding, encode, nagalakshmi2008transcriptional, 
tang2009mrna, banfai2012long, mortazavi2008mapping, wang2008alternative, 
katz2010analysis, deng2011isoform, lu2010function, mercer2011targeted, 
howald2012combining, lalonde2011rna}

\section{RNA-Seq 数据分析方法简介}

\section{现有 RNA-Seq 数据定量分析方法}


