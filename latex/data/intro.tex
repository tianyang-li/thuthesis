\chapter{引言}

\section{RNA-Seq}
\nocite{wang2009rna, ozsolak2010rna}

RNA-Seq 是对 RNA 序列进行测量的一种技术, 
它是近几年来发展起来的通过深度测序用于研究转录组的一种技术. 
与其他的方法相比起来, RNA-Seq 揭露了生物的转录组中更多的复杂性. 
同时, RNA-Seq 能够更好地研究生物的转录组中各转录本的表达量. 
通过了解细胞中在某种特定条件下转录组中的转录本的组成, 
以及每一个转录本的表达量, 我们可以了解基因上的不同的功能模块, 
进而了解生物的发育过程, 以及疾病与人体之间的关系. 
通过使用 RNA-Seq, 我们已经对编码蛋白质的基因以及它们的剪切异构体 (isoform) 有了更为深入的了解. 
此外, RNA-Seq 也帮助我们对于基因上的非编码区域有了更为深入的认识, 
例如 lncRNA (long non-coding RNA).
并且我们对 sRNA (small RNA), microRNA 等也有了更全面的了解. 
\cite{pickrell2010understanding, encode, nagalakshmi2008transcriptional, 
tang2009mrna, banfai2012long, mortazavi2008mapping, wang2008alternative, 
katz2010analysis, deng2011isoform, lu2010function, mercer2011targeted, 
howald2012combining, lalonde2011rna, djebali2012landscape, 
derrien2012gencode, gerstein2012architecture, fairfax2012genetics, 
morrissy2011extensive, howald2012combining, park2012rna, 
tilgner2012deep, orom2010long, mercer2011human, chung2011computational, 
gingeras2009implications, roy2010identification, axtell2011vive, 
berezikov2010evolutionary, cherbas2011transcriptional, anders2012detecting, 
stoeckius2009large, lau2009abundant}

在 RNA-Seq 技术出现之前, 人们主要通过微阵列 (microarray) 对转录组进行定量分析和研究 \cite{schena1995quantitative}. 
但是与 RNA-Seq 相比, 微阵列存在若干问题 \cite{wang2009rna}: 
\begin{itemize}
\item 微阵列只能在序列已知的情况下使用

\item 结果会受交叉杂交 (cross-hybridization) 影响 \cite{okoniewski2006hybridization, royce2007toward}

\item 测量的表达量范围有限
\end{itemize}
另外, 在 RNA-Seq 技术出现之前, 人们主要通过 Sanger 测序法对互补 DNA (cDNA) 序列或者表达序列标签库 (EST libraries) 进行测序来研究转录组的序列 \cite{boguski1994gene, gerhard2004status}. 
但是 Sanger 测序价格昂贵, 同时测序通量偏低, 无法对转录组进行定量分析和研究. 
高通量测序技术的发展使得我们能够在较短的时间内用较少的成本对大量序列进行测序, 
同时建立测序数量和实际被测序分子的数量间的关系. 
这是 RNA-Seq 在当今被广泛应用的一个主要原因. 

\section{RNA-Seq 数据分析方法简介}

\section{现有 RNA-Seq 数据定量分析方法}


