\chapter{文献翻译}

\section{文献索引}
Garber, Manuel, et al. ``Computational methods for transcriptome annotation and quantification using RNA-seq.'' Nature Methods 8.6 (2011): 469-477.

\section{翻译}
高通量 RNA 测序(RNA-Seq)帮助我们对转录有一个全面的了解,从而完整的注释和定量的样本之间的基因。实现这一目标,需要越来越复杂的计算方法。这些计算的挑战可分为三大类:(一)读段比对,(二)转录重建及(三)表达量化。在这里,我们解释的主要概念和实践的挑战,和为每个类别一般类的解决方案。最后,我们强调这些类之间的相互依存和讨论不同生物应用的好处。

定义的精确地研究所有基因在不同的细胞类型和表达是理解生物学的关键。直到最近,这样的数据是非常昂贵的生产和实验费力。注释转录的主要方法,需要缓慢而昂贵的过程中克隆的 cDNA 或表达序列标签(EST)库,然后由毛细管测序。

DNA 测序技术的最新进展使人们有可能来自细胞的 RNA 序列的 cDNA 的大规模并行测序技术,这一过程被称为 RNA-Seq。

在这里,我们重点介绍 RNA-Seq 分析面临的核心挑战和需要解决的计算方法。首先,我们描述对齐的方法直接读段到一个参考转录或基因组('读段比对')。第二,我们讨论的方法来确定表达的基因和亚型('转录重建')。第三,我们提出了基因及变异体丰度估计的方法,以及用于分析样品('表达定量')之间的差异表达的方法。

由于 RNA-Seq 的数据生成的持续改善,有不同的成熟可用的计算工具。在某些领域,如读段比对,丰富的算法存在,但在其他方面,如差异表达分析,解决方案才刚刚开始出现。我们并不较全面地描述每一种方法,而是突出重点的共同原则以及相关 RNA-Seq 分析中每种方法及其应用的关键差异。我们还讨论了这些不同的方法如何可能会影响数据的结果和解释。虽然我们讨论每三个类别作为独立的单位,RNA-Seq 的数据分析往往需要从所有三个类别使用方法。这里描述的方法在很大程度上是独立选择库建设协议,同时我们介绍双端测序(读从两端的一个片段),它提供了有价值的信息,在 RNA-Seq 的分析各个阶段有显着的应用。

\subsection{比对短 RNA-Seq 读段}

RNA-Seq 分析中最基本的任务之一是将读段比对到基因组参考序列上。序列是生物信息学中的一个经典问题, 并且有多种解决方案为 EST 进行比对。

比对没有剪切的读段到基因组参考序列对于定量分析有十分重要的作用。

没有剪切的读段仅限于识别已知的外显子和剪切位点,不允许涉及新外显子的剪接事件的识别。另外,读段可以对齐到整个基因组,其中包括内含跨越读段,可能会有很大的差距,需要妥善安置。有几种方法,统称为“拼接对准',落入两大类:'第一外显子'和'种子和延长。第一外显子也支持配对末端读比对,从而增加了对准特异性。

第一外显子的方法更快,需要更少的计算资源相比,种子扩展方法。例如,种子扩展方法 (GSNAP) 需要更长的时间 ~8x (~340个CPU小时) 比第一个外显子的方法(顶帽)~1.5x 拼接读段。然而,这些额外的剪接点的生物学意义还没有被证实。

第一外显子的方法可能发生基因有重组的错误从而错过拼接路线。与此相反,种子扩展方法评价剪接和未拼接的路线在同一步骤中,这降低偏向未拼接的路线,得到每次读操作的最佳位置。种子扩展方法比第一外显子的方法比对读段时多态性物种时有更好的表现。

\subsection{转录组重建}

精确地比对并且通过一个特定样本组装读段成转录单位。总的来说,我们这个过程称为转录重建。转录组重建是一项困难的计算任务,主要有三个原因。首先,基因表达跨越几个数量级,表示仅由少数读段某些基因。其次,读段源于成熟的 mRNA(外显子),以及来自不完全剪接的前体 RNA(包含内含子序列),使得难以识别的成熟转录本。第三,读段很短,基因可以有许多亚型,它具有挑战性,以确定哪些亚型产生每个读。

有几种方法来重建转录,它们分为两大类:“基因组引导'和'基因组独立'。基因引导的方法依赖于一个参考基因组首先比对所有读段的基因组,然后组装重叠读入笔录。相比之下,独立于基因组的方法组装成的转录本,而无需使用参考基因​​组直接读段。

基因引导重建。现有基因制导方法可分为两大类:'外显子鉴定“和”基因制导汇编“的方法。

外显子识别早期开发时,读段短(〜36个碱基)和几个外显子 - 外显子路口对齐。他们首先定义假定外显子覆盖的岛屿,然后使用拼接读段这些覆盖岛屿定义跨越外显子边界和外显子之间建立连接。外显子识别方法提供转录本重建问题最合适的短读段第一个办法来解决,但他们是动力不足,以确定低表达,长和选择性剪接基因的全长结构。

要利用更长的读长,基因组制导组件方法,如袖扣。

经文和袖扣也有类似的计算要求,既可以在个人计算机上运行。组装类似的转录本,在高表达水平,但显着差异表达的转录袖扣报告 3x 更多的位点(70,000与25,000元),其中大部分不及格经文所使用的统计学意义阈值较低。在最极端的情况下,超过300个亚型为一个单一的轨迹,而袖扣的经文报告报告相同基因的11种亚型。

基因无关重建。而非比对读段参考序列,基因组独立的转录重建算法如transAbyss。重叠的k  -  1之间的基地这k个聚体构成的曲线图中,可以构造所有可能的序列。接下来,在图中遍历路径,指导阅读和配对末端覆盖水平,消除虚假的分支点引入K-聚体所共享的不同转录,由读段和配对两端的,但不支持。所有剩余的路径,然后通过图形报告​​作为一个单独的转录本。

虽然基因组独立重建的概念很简单,有两个主要的并发症:区别序列变异的错误,并找到灵敏度和图形的复杂性之间的最佳平衡。比对的策略不同,测序错误介绍分支点在图中,增加其复杂性。要消除这些文物,基因组独立的方法在不同的路径图中的覆盖面和覆盖截止申请来决定何时遵循的路径,或将其删除时。

为了配合转录丰度的变化,内在的表达谱数据的几种方法,如 TransABySS,使用变 k-mer 的战略来获得权力,表达水平之间组装的转录本,尽管在CPU供电的费用和需要并行执行。

相比重建战略。基因组引导和基因组独立的算法已上报准确地重建数千转录本和许多替代剪接形式,对上述问题的答案变得不那么清晰,取决于分析目标上。在许多情况下,一种混合​​的方法结合了基因组独立和基因组制导的战略可能最适合捕捉已知信息,以及把握新的变化。在实践中,基因组独立的方法需要相当大的计算资源(〜650的CPU小时和16千兆字节的随机存取存储器(RAM))相比,基因组的引导方法(〜4的CPU时间和<4千兆字节RAM)。

\subsection{估计转录表达量}

表达定量一直是一个重要的应用。在过去的十年中,DNA微阵列技术高通量转录组分析的首选技术。当使用RNA-seq的估计基因表达,阅读计数需要得到适当的标准化提取有意义的表达估计。

考虑到这些问题,读段每碱基每百万比对读段(RPKM)度量的转录本,转录本的读段计数正常化它的长度和总数的比对读段样品中。

由于许多基因有多种亚型,其中许多股票外显子,和许多基因家族有密切的旁系,一些读段不能被明确分配的转录本。

我们注意到,一些潜在的亚型极大地影响结果,引入不确定性与不正确或拼接错误的异构体。因此,工作时,产生最大的异构体集的方法,它是必要的预过滤转录之前,某些基因的表达估计。这适用于两个基因组的引导下,以及基因组独立的算法。

通常情况下,目标是要估计每个基因的表达,而不是为每个异构体或转录本,并且,如下面所讨论的交点,可以降低功耗的差异表达分析。

\subsection{RNA-Seq 的差异表达分析}

量化和规范化的表达式的值,一个重要的问题是要了解如何将这些表达水平的不同而不同的条件。过去的十年中看到了发展广泛的方法表达差异进行统计分析,使用微阵列。

为了适应基于计数性质的 RNA-Seq 的数据,最初的方法为蓝本所观察到的读段使用计数型分布,如泊松分布。

重要的是要注意,虽然这些方法可以分配差异表达的意义,生物的结论,必须小心诠释。例如,虽然在测序过程的可变性比较低的微阵列杂交和最重要的,因为在生物样品中的内在变异。至于与任何生物测量,生物复制提供内在的,非技术性的转录表达变异的唯一衡量和因而是有史以来差异表达分析的关键。

对基因的差异表达分析的量化策略。当执行差异表达分析,大多数方法采取输入归一读的计数每个基因在每个条件。使用简化基因定量模型,如外显子交会法或外显子工会法,可能会导致意想不到的结论。当一个基因有多个异构体,基因的表达的变化可能不会导致相应的变化,原始的基因水平计数。考虑两种亚型基因的情况下。

\subsection{结论和预期未来的发展}

随着测序技术的成熟,现有的计算工具将需要发展以满足新的要求,新的工具,会出现使新的应用。例如,读段长度的不断增加,新的比对方法将需要有效地使数以百万计的长读段一个艰巨的任务。更长的读段经常跨越多个外显子 - 外显子路口,转录本重建和定量分析方法,将有利于纳入更完整的异构信息编码较长读段。标准RNA-seq的方法不适合注释5'起始位点和3'端的转录本通过使用专门的RNA-seq的库,确定两端转录重建方法,将改善转录本注解。从RNA-seq的数据估算表达方法需要改进,以更好地处理日益普及生物复制实验和理想模式(自动减去)系统的误差来源所引入实验室方法(如3'-端偏见)。提供的序列表达的转录,RNA-seq的编码信息等位​​基因变异和RNA加工,所以重建方法应适应这种变异占报告。计算技术的改进,既可以在实验室以及不断循环,将继续扩大的RNA-seq的可能性,使得这项技术适用于更多种类的生物学问题。





