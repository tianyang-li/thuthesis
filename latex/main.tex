%%% Local Variables:
%%% mode: latex
%%% TeX-master: t
%%% End:

\documentclass[bachelor,nofonts]{thuthesis}
%\documentclass[master]{thuthesis}
%\documentclass[doctor]{thuthesis}
% \documentclass[%
%   bachelor|master|doctor|postdoctor, % mandatory option
%   winfonts|nofonts|adobefonts, % mandatory only for bachelor and Linuxer
%   secret,
%   openany|openright,
%   arialtoc,arialtitle]{thuthesis}
% 当使用 XeLaTeX 编译时,本科生、Linux 用户需要加上 nofonts 选项;
% 当使用 PDFLaTeX 编译时,adobefonts 选项等效于 winfonts 选项(缺省选项)。

% 所有其它可能用到的包都统一放到这里了,可以根据自己的实际添加或者删除。
\usepackage{thutils}

% 你可以在这里修改配置文件中的定义,导言区可以使用中文。
% \def\myname{薛瑞尼}

\begin{document}

% 定义所有的eps文件在 figures 子目录下
\graphicspath{{figures/}}


%%% 封面部分
\frontmatter

%%% Local Variables:
%%% mode: latex
%%% TeX-master: t
%%% End:
\secretlevel{} \secretyear{}

\ctitle{通过 RNA-Seq 估计转录本长度和辨识剪切异构体的研究}
% 根据自己的情况选,不用这样复杂
\makeatletter
\ifthu@bachelor\relax\else
  \ifthu@doctor
    \cdegree{工学博士}
  \else
    \ifthu@master
      \cdegree{工学硕士}
    \fi
  \fi
\fi
\makeatother


\cdepartment[自动化]{自动化系}
\cmajor{自动化}
\cauthor{李天阳} 
\csupervisor{张学工}
% 如果没有副指导老师或者联合指导老师,把下面两行相应的删除即可。
\cassosupervisor{江瑞}
% 日期自动生成,如果你要自己写就改这个cdate
%\cdate{\CJKdigits{\the\year}年\CJKnumber{\the\month}月}

% 博士后部分
% \cfirstdiscipline{计算机科学与技术}
% \cseconddiscipline{系统结构}
% \postdoctordate{2009年7月——2011年7月}

\etitle{Research on using RNA-Seq to estimate transcript length and identify isoforms} 
% 这块比较复杂,需要分情况讨论:
% 1. 学术型硕士
%    \edegree:必须为Master of Arts或Master of Science(注意大小写)
%              “哲学、文学、历史学、法学、教育学、艺术学门类,公共管理学科
%               填写Master of Arts,其它填写Master of Science”
%    \emajor:“获得一级学科授权的学科填写一级学科名称,其它填写二级学科名称”
% 2. 专业型硕士
%    \edegree:“填写专业学位英文名称全称”
%    \emajor:“工程硕士填写工程领域,其它专业学位不填写此项”
% 3. 学术型博士
%    \edegree:Doctor of Philosophy(注意大小写)
%    \emajor:“获得一级学科授权的学科填写一级学科名称,其它填写二级学科名称”
% 4. 专业型博士
%    \edegree:“填写专业学位英文名称全称”
%    \emajor:不填写此项
\edegree{Bachelor of Engineering} 
\emajor{Automation} 
\eauthor{Li Tianyang} 
\esupervisor{Zhang Xuegong} 
\eassosupervisor{Jiang Rui} 
% 这个日期也会自动生成,你要改么?
% \edate{December, 2005}

% 定义中英文摘要和关键字
\begin{cabstract}
	RNA-Seq 是最近几年发展起来的通过高通量测序对转录组中的序列进行测序的一种技术。 
	RNA-Seq 技术的发展使得人们在最近几年当中对于生物中的基因表达的规律, 
	以及基因组上的功能模块有了更为深入的了解。 
	在通过 RNA-Seq 数据确定基因的表达量时, 我们需要知道基因序列的长度。 
	但是在没有基因注释或者没有基因组参考序列时, 我们需要一种得知基因的长度的方法。
	本文提出了一个通过 RNA-Seq 数据对转录本的长度进行估计的统计方法。 
	通过该方法, 我们可以在基因组参考序列没有基因注释信息, 以及没有基因组参考序列, 
	的情况下使用 RNA-Seq 数据估计出转录本的长度。
	同时, 在 RNA-Seq 数据中我们发现读段的分布位置不均匀。
	此处我们对 RNA-Seq 数据中读段分布的不均匀性做了初步的分析。
	此外, 真核生物的基因在有多个外显子的情况下会有选择性剪切的现象发生, 
	同一个基因可能会产生多个剪切异构体。
	通过 RNA-Seq 数据我们可以辨别一个基因的不同的剪切异构体。
	本文证明了用最大似然的方法通过真核生物 RNA-Seq 数据辨识基因的剪切异构体是一个 NP 难问题。
\end{cabstract}

\ckeywords{RNA-Seq, 转录组, 转录本}

\begin{eabstract} 
	RNA-Seq is a technology developed in the last few years for sequencing the transcriptome using high throughput sequencing. 
	Using RNA-Seq, people have gained much deeper understanding of gene expression patterns, 
	and functional modules in genomes. 
	When estimating a transcript's expression level with RNA-Seq, 
	we need to know the length of the transcript's sequence. 
	However, when no annotations or reference genome sequences are available, 
	we need another method to know the transcript's length. 
	Here, we present a statistical method to estimate transcript length using RNA-Seq. 
	Using this method, we can estimate a transcript's length when no annotations are available for the reference genome sequences, or when the reference genome sequences are not available. 
	We also observed that RNA-Seq reads are non-uniformly distributed. 
	Here, we present a preliminary analysis on the non-uniform distribution of RNA-Seq reads. 
	And it has been observed in eukaryotes a gene with multiple exons can correspond to multiple isoforms due to alternative splicing. With RNA-Seq, we can determine a gene's isoroms. 
	Here, we prove that using eukaryotic RNA-Seq data to identify a gene's isoforms by maximum likelihood is NP-hard.
\end{eabstract}

\ekeywords{RNA-Seq, transcriptome, transcript}




% 设置 PDF 文档的作者、主题等属性
\makeatletter
\thu@setup@pdfinfo
\makeatother
\makecover

% 目录
\tableofcontents

% 符号对照表
\begin{denotation}

\item[HPC] 高性能计算 (High Performance Computing)

\end{denotation}



%%% 正文部分
\mainmatter
\include{data/chap01}
\include{data/chap02}


%%% 其它部分
\backmatter

% 本科生要这几个索引,研究生不要。选择性留下。
\makeatletter
\ifthu@bachelor
  % 插图索引
  \listoffigures
  % 表格索引
  \listoftables
  % 公式索引
  \listofequations
\fi
\makeatother


% 参考文献
\bibliographystyle{thubib}
\bibliography{ref/refs}


% 致谢
%%% Local Variables:
%%% mode: latex
%%% TeX-master: "../main"
%%% End:

\begin{ack}
	衷心感谢导师张学工教授和江瑞副教授对本人的精心指导。 
	
	同时也感谢 \href{https://github.com/xueruini/thuthesis}{\thuthesis}, 
	以及其他各种开源项目给予我的帮助和支持。 
\end{ack}


% 附录
\begin{appendix}
%%% Local Variables: 
%%% mode: latex
%%% TeX-master: "../main"
%%% End: 

\chapter{源代码}
\begin{itemize}
\item \url{https://github.com/tianyang-li/de-novo-rna-seq-quant-1}
\item \url{https://github.com/tianyang-li/thu-undegrad-thesis-code}
\item \url{https://github.com/tianyang-li/aarsa}
\item \url{https://github.com/tianyang-li/rna-seq-len-est-0}
\item \url{https://github.com/tianyang-li/misc-bioinfo-0}
\item \url{https://github.com/tianyang-li/de-novo-metatranscriptome-analysis--the-uniform-model}
\item \url{https://github.com/tianyang-li/human-rna-seq-analysis-0}
\item \url{https://github.com/tianyang-li/de-novo-rna-seq-quant-with-contigs-py-0}
\item \url{https://github.com/tianyang-li/bi-misc}
\item \url{https://code.google.com/p/meta-transcriptome/}
\end{itemize}


\end{appendix}

% 个人简历
\begin{resume}

  \resumeitem{个人简历}

  %xxxx 年 xx 月 xx 日出生于 xx 省 xx 县. 
  
  2009 年 8 月考入清华大学自动化系自动化专业, 2013 年 7 月本科毕业并获得工学学士学位。

  \resumeitem{发表的学术论文} % 发表的和录用的合在一起

  \begin{enumerate}[{[}1{]}]
	\item T. Li, R. Jiang, and X. Zhang. 
	Isoform reconstruction using short RNA-Seq reads by maximum likelihood is NP-hard. 
	ArXiv e-prints, May 2013. \url{http://arxiv.org/abs/1305.0916}.

	%\item Tianyang Li, Fuye Han, Shuai Ding, and Zhen Chen. 
	%LARX: Large-Scale Anti-Phishing by Retrospective Data-Exploring Based on a Cloud Computing Platform. 
	%In Computer Communications and Networks (ICCCN), 2011 
	%Proceedings of 20th International Conference on, 2011.
	
	\item Tianyang Li, Rui Jiang and Xuegong Zhang. 
	\textit{De novo} transcript reconstruction and abundance estimation in eukaryotic RNA-Seq data analysis. 
	RECOMB 2013. (Poster)
  \end{enumerate}
  
\end{resume}

\end{document}
